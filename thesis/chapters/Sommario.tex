\thispagestyle{plain}
\setcounter{page}{4}

La latenza di coda compromette le prestazioni dei moderni microservizi. Le tecniche di mitigazione tradizionali, come l'hedging delle richieste delle applicazioni, soffrono di instabilità (jitter) dovuta alla pianificazione del sistema operativo e alla garbage collection.
Questa tesi propone un'implementazione trasparente dell'hedging delle richieste basata su eBPF, che sposta la logica di controllo dallo spazio utente al kernel Linux. Il sistema intercetta il traffico UDP e gestisce le ritrasmissioni senza richiedere modifiche al codice sorgente.
I test condotti in un ambiente caotico deterministico dimostrano una riduzione del P99 da 400 ms a 13 ms. Rispetto alle soluzioni applicative, l'approccio del kernel elimina le fluttuazioni temporali, garantendo una stabilità di esecuzione superiore con un overhead trascurabile.
